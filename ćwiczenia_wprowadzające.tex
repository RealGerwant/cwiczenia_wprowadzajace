%% Wz�ra sprawozdania w LateXu

\documentclass[polish,polish,a4paper]{article}
\usepackage[T1]{fontenc}
\usepackage[cp1250]{inputenc}
\usepackage{babel}
\usepackage{pslatex}
\usepackage{pgfplots}
\usepackage{circuitikz} 
\usepackage{setspace}
\usepackage{caption}
%\usetikzlibrary{circuits.ee.IEC}
\usepackage{anysize}
\marginsize{2.5cm}{2.5cm}{3cm}{3cm}

\newcommand{\PRzFieldDsc}[1]{\sffamily\bfseries\scriptsize #1}
\newcommand{\PRzFieldCnt}[1]{\textit{#1}}
\newcommand{\PRzHeading}[8]{
%% #1 - nazwa laboratorium
%% #2 - kierunek 
%% #3 - specjalno�� 
%% #4 - rok studi�w 
%% #5 - symbol grupy lab.
%% #6 - temat 
%% #7 - numer lab.
%% #8 - sk�ad grupy �wiczeniowej

\begin{center}
\begin{tabular}{ p{0.32\textwidth} p{0.15\textwidth} p{0.15\textwidth} p{0.12\textwidth} p{0.12\textwidth} }

  &   &   &   &   \\
\hline
\multicolumn{5}{|c|}{}\\[-1ex]
\multicolumn{5}{|c|}{{\LARGE #1}}\\
\multicolumn{5}{|c|}{}\\[-1ex]

\hline
\multicolumn{1}{|l|}{\PRzFieldDsc{Kierunek}}	& \multicolumn{1}{|l|}{\PRzFieldDsc{Specjalno��}}	& \multicolumn{1}{|l|}{\PRzFieldDsc{Rok studi�w}}	& \multicolumn{2}{|l|}{\PRzFieldDsc{Symbol grupy lab.}} \\
\multicolumn{1}{|c|}{\PRzFieldCnt{#2}}		& \multicolumn{1}{|c|}{\PRzFieldCnt{#3}}		& \multicolumn{1}{|c|}{\PRzFieldCnt{#4}}		& \multicolumn{2}{|c|}{\PRzFieldCnt{#5}} \\

\hline
\multicolumn{4}{|l|}{\PRzFieldDsc{Temat Laboratorium}}		& \multicolumn{1}{|l|}{\PRzFieldDsc{Numer lab.}} \\
\multicolumn{4}{|c|}{\PRzFieldCnt{#6}}				& \multicolumn{1}{|c|}{\PRzFieldCnt{#7}} \\

\hline
\multicolumn{5}{|l|}{\PRzFieldDsc{Sk�ad grupy �wiczeniowej oraz numery indeks�w}}\\
\multicolumn{5}{|c|}{\PRzFieldCnt{#8}}\\

\hline
\multicolumn{3}{|l|}{\PRzFieldDsc{Uwagi}}	& \multicolumn{2}{|l|}{\PRzFieldDsc{Ocena}} \\
\multicolumn{3}{|c|}{\PRzFieldCnt{\ }}		& \multicolumn{2}{|c|}{\PRzFieldCnt{\ }} \\

\hline
\end{tabular}
\end{center}
}

\begin{document}

\PRzHeading{Laboratorium Podstaw Elektroniki}{Informatyka}{--}{I}{nie mamy}{�wiczenia wprowadzaj�ce}{1}{Piotr Wi�tczak(132339), Robert Ciemny(136693), Kamil Basiukajc(136681)}{}


\section{�wiczenia wprowadaj�ce}
\subsection{Rezystory}
\begin{spacing}{1,5}

\captionof{table}{Warto�ci odczyt�w i pomiar�w rezystancji}
\[
\begin{array}{|c|c|c|c|}
\hline
R&Barwa/oznaczenia&Odczyt&Pomiar \\ 
\hline
R_{1}&--&--&-- \\ 
\hline
R_{2}&--&--&-- \\ 
\hline
R_{3}&--&--&-- \\ 
\hline
R_{4}&--&--&-- \\ 
\hline
R_{5}&--&--&-- \\ 
\hline
R_{6}&--&--&-- \\ 
\hline
\end{array}
\]

\subsection{Kondensatory}

	
	\captionof{table}{Warto�ci odczyt�w i pomiar�w pojemno�ci}
	\[
	\begin{array}{|c|c|c|c|}
	\hline
	C&Oznaczenia&Odczyt&Pomiar \\ 
	\hline
	C_{1}&--&--&-- \\ 
	\hline
	C_{2}&--&--&-- \\ 
	\hline
	C_{3}&--&--&-- \\ 
	\hline
	C_{4}&--&--&-- \\ 
	\hline
	C_{5}&--&--&-- \\ 
	\hline
	C_{6}&--&--&-- \\ 
	\hline
	\end{array}
	\]
	
	
\subsection{Cewki}

\captionof{table}{Warto�ci odczyt�w i pomiar�w indukcyjno�ci}
\[
\begin{array}{|c|c|}
\hline
L&Pomiar \\
\hline
L_{1}&-- \\
\hline
\end{array}
\]

\section{Obwody}
\subsection{Obliczanie rezystancji zast�pczej}

\begin{figure}[!h]
	\centering
	\begin{circuitikz}[scale=1,european]
		
		\draw
		(0,0) node[anchor=west] {A}
		to[short, o-] (-3,0) 
		to (-3,3) --
		(-3,3) to [european resistor, l=$R_{7}$, a=$1k \Omega$] (-1,3) --
		(-1,4) to (-1,2) --
		(-1,4) to[european resistor, l=$R_{5}$, a=$100 \Omega$] (1,4)
		(-1,2) to[european resistor, l=$R_{6}$, a=$200 \Omega$] (1,2)
		(1,4) to (1,2) 
		(1,3) to (1.5,3)
		(1.5,2) to (1.5,7)
		(1.5,2) to [european resistor, l=$R_{4}$, a=$270 \Omega$] (3.5,2)
		(1.5,4) to [european resistor, l=$R_{3}$, a=$1k \Omega$] (3.5,4)
		(1.5,5.5) to [european resistor, l=$R_{2}$, a=$3k \Omega$] (3.5,5.5)
		(1.5,7) to [european resistor, l=$R_{1}$, a=$2k \Omega$] (3.5,7)
		(3.5,7) to (3.5,2)
		(3.5,3) to (4,3)
		(4,4) to[european resistor, l=$R_{8}$, a=$1 \Omega$] (6,4)
		(4,2) to[european resistor, l=$R_{9}$, a=$100 \Omega$] (6,2)
		(4,4) to (4,2)
		(6,4) to (6,0)--
		(2,0) node[anchor=east] {B}
		to[short ,-o]  (2,0);
	\end{circuitikz}
	\caption{Obw�d rezystancyjny}
	\label{fig:rlc}
\end{figure}



\begin{figure}[!h]
	\centering
	\begin{circuitikz}[scale=1,european]
		
		\draw
		(0,0) node[anchor=west] {A}
		to[short, o-] (0,0.5) 
		(0,0.5) to (3,0.5)
		(3,0.5) to [european resistor, l=$R_{2}$, a=$2k\Omega$] (5,0.5)
		(5,0.5) to (5,-1.5)
		(5,-1.5) to [european resistor, a=$R_{3}$,l=$2k\Omega$] (3,-1.5)
		to (0,-1.5)
		(0,-1)node[anchor=east] {B}
		to [short, o-] (0,-1.5)
		(2,0.5) to [european resistor, l=$1k\Omega$, a=$R_{1}$] (2,-1.5);

	\end{circuitikz}
	\caption{(a)}
	\label{fig:rlc}
\end{figure}


\begin{figure}[!h]
	\centering
	\begin{circuitikz}[scale=1,european]
		
		\draw
		(0,0) to [european resistor, l=$R_{1}$, a = $1k\Omega$] (0,2)
		(0,2) to (2,2)
		to [european resistor , l=$R_{3}$,a=$1k\Omega$] (5,2)
		to [european resistor, a=$ R_{5} $, l = $ 100\Omega $] (5,0)
			to [european resistor, a=$R_{4}$, l=$100\Omega$] (2,0)
			to (0,0)
		(2,2) to [european resistor, l= $2k\Omega$, a= $R_{2}$] (2,0)
		(2.5,-1) node[anchor = west] {A}
		to [short, o-] (2.5,0) 
		(4.5,-1) node[anchor = east] {B}
		to [short, o-] (4.5,0);
	\end{circuitikz}
	\caption{(b)}
	\label{fig:rlc}
\end{figure}


\begin{figure}[!h]
	\centering
	\begin{circuitikz}[scale=1,european]
		
		\draw
	(0,-0.5) node[anchor = north] {A}
	to [short, o- ] (0,0)
	(0,0) to (2,0) 
	to [european resistor, l=$R_{4}$, a=$100\Omega$] (4,0)
	to (6,0)
	(2,0) to (2,2)
	to [european resistor, l=$R_{4}$, a = $1k\Omega$] (4,2)
	to (4,0)
	(5,0) to [european resistor, a = $ R_{2} $, l = $ 2k\Omega $] (5,-2)
	to (6,-2)
	(1,0) to [european resistor, a= $R_{1}$, l=$2k\Omega$] (1,-2)
	to (0,-2)
	to [short, -o] (0,-1.5)
	(0,-1.5) node [anchor = south] {B};
	\end{circuitikz}
	\caption{(c)}
	\label{fig:rlc}
\end{figure}



\end{spacing}
\end{document}


